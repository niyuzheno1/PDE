\documentclass{article}
\usepackage[utf8]{inputenc}

\title{PDE}
\author{waluigi120 }
\date{April 2018}

\begin{document}

\maketitle

\section{First-order Linear Equation}

\subsection{A simplest example}
$$\frac {\partial u} {\partial x} = 0$$
$$\Rightarrow <1,0> \cdot <u_x,u_y> = 0$$
$$\Rightarrow y = c$$

\begin{equation}\label{eq1}
u(x,y) = f(y)
\end{equation}
where f is arbitrary


\subsection{The constant coefficient equation}

Definition:
$$au_x+bu_y=0, a^2+b^2 > 0$$

Solutions:

\subsubsection{Geometric Method}

$$<a,b> \cdot <u_x,u_y> = 0$$

So $<u_x,u_y>$ is orthogonal to $<a,b>$. Since $<u_x,u_y>$ is the gradient of $u$, the function $u$ along the direction $<a,b>$ should be constant.  $bx-ay$ remains constant. It is named as characteristic line.

Similar to $(1)$, $u = f(bx-ay)$

\subsubsection{Coordinate Method}

Change variables 

\begin{equation} 
 x' = ax+by, y' = bx-ay
\end{equation}

Then by chain rule,
$$u_x = \frac{\partial u}{\partial x} = \frac{\partial u}{\partial x'} \frac{\partial x'} {\partial x} + \frac{\partial u}{\partial y'} \frac{\partial y'} {\partial x}  = a u_{x'} + b u_{y'}$$

Similiarily,
$$u_y = bu_{x'} - a u_{y'}$$

Obviously by the constant coefficient equation,
$$au_x+bu_y=(a^2+b^2)u_{x'} = 0$$

Therefore, the problem is reduced to 
$$u_{x'} = 0$$

\subsubsection{Problems}
1. Solve the PDE $4u_x - 3 u_y = 0$ with condition $u(0,y) = y^3$

Solution:

$<4,-3> \cdot <u_x.u_y> = 0 \rightarrow 3x+4y = c \rightarrow u(x,y) = f(3x+4y)$ 

$u(0,y) = y^3 \rightarrow u(0,y) = f(4y)=y^3,f(u)=(\frac{u}{4})^3$

Overall, $f(3x+4y) = \frac{(3x+4y)^3}{64}$

\subsection{The variable coefficient equation}
Definition

\begin{equation} \label{eq2}
u_x+yu_y =0
\end{equation}


It is linear (no $u^2$) and homogeneous (right hand side is zero).

One way to solve it:

Since the gradient's dot product with $<1,y>$ is zero, $<1,y>$ is tangent to the curve. 

$$\frac{dy}{dx} = y \rightarrow y = Ce^x$$

$y = Ce^x$ is called characteristic curve

Similar to the constant coefficient equation, the solution to the problem is $u(x,y) = f(C) = f(ye^{-x})$

\subsubsection{Problems}

1.Solve $u_x+2xy^2u_y=0$

Solution:

$<1,2xy^2> \cdot <u_x,u_y>=0 \rightarrow \frac{dy}{dx} = 2xy^2 \rightarrow \frac{dy}{y^2} = 2x dx \rightarrow -y^{-1} = x^2 + C \rightarrow C = -y^{-1}-x^2$

So $u(x,y) = f(y^{-1}+x^2)$

\subsection{Problems of section 1}

1. Solve $3u_y+u_{xy} = 0$

Solution:

Let $v = u_y$, 

$3v + v_x = 0 \rightarrow \frac{dv}{dx} = -3v \rightarrow v = f(y) e^{-3x}$

$u = g(y)e^{-3x} + h(x)$

2. Solve $\sqrt{1-x^2} u_x + u_y =0$, whose initial condition is $u(0,y) = y$

$\sqrt{1-x^2} u_x + u_y =0 \rightarrow <\sqrt{1-x^2},1> \cdot \nabla u = 0 \rightarrow \frac{dy}{dx} = \frac{1}{\sqrt{1-x^2}}$

$$ \int \frac{1}{\sqrt{1-x^2}} dx$$

Let $x = sin(\theta), dx = cos(\theta) d\theta, 1-x^2 = cos^2(\theta)$

$$ \int \frac{1}{cos(\theta)} cos(\theta) d\theta = \arcsin(x)$$

$\frac{dy}{dx} = \frac{1}{\sqrt{1-x^2}} \rightarrow y = \arcsin(x) + C \rightarrow u(x,y) = f(y-\arcsin(x))$

$u(0,y) = y \rightarrow u(x,y) = y-\arcsin(x)$

3.
https://math.stackexchange.com/questions/729165/finding-the-uniquely-determined-region-of-a-pde

4.Solve $au_x+bu_y +c u =0$

$au_x/u+bu_y/u +c  =0$

Let $v = \ln(u),$

$av_x+bv_y +c  =0$

Let $x' = ax+by, y' = bx-ay$

$v_x = av_{x'}+bv_{y'}, v_y = bv_{x'} -av_{y'}$

So $av_x+bv_y +c  =0 \rightarrow (a^2+b^2)v_{x'} + c =0 \rightarrow v = \frac{-cx'}{a^2+b^2} + g(bx-ay)$

$u = f(bx-ay) e^{\frac{-c(ax+by)}{a^2+b^2}}$

5. Solve $u_x+u_y+u = e^{x+2y}, u(x,0)=0$

$$\frac{dx}{1} = \frac{dy}{1} = \frac{du}{e^{x+2y}-u}$$

$\frac{dx}{1} = \frac{dy}{1} \rightarrow y=x+C$

$$\frac{du}{e^{x+2y}-u} = \frac{dx}{1}\rightarrow  \frac{du}{dx} = e^{x+2y} -u \rightarrow \frac{du}{dx} + u =  e^{x+2y} \rightarrow e^{x}\frac{du}{dx} + e^{x}u = e^{2x+2y}$$
$$\rightarrow \frac{e^{x} du}{dx} =   e^{2x+2(x+C)} \rightarrow {e^{x} u} =   \frac{1}{4}e^{2x+2y}+ C_1 = {e^{x} u} =   \frac{1}{4}e^{2x+2y}+ f(y-x) $$

$$ u = \frac{1}{4}e^{x+2y} + \frac{f(y-x)}{e^x}$$

By initial condition

$u(x,0) = \frac{1}{4}e^{x} + \frac{f(-x)}{e^x} = 0 \rightarrow f(x) = -\frac{1}{4}e^{-2x}$

$u = \frac{1}{4}e^{x+2y} - \frac{1}{4}e^{x-2y}$

6. Solve $au_x+bu_y = f(x,y)$

$$\frac{dx}{a} = \frac{dy}{b} = \frac{du}{f(x,y)}$$

$\frac{dx}{a} = \frac{dy}{b} \rightarrow bx = ay+C \rightarrow x = \frac{a}{b} y + C$

$\frac{dy}{b} = \frac{du}{f(x,y)} \rightarrow du  = \frac{f(\frac{a}{b} y + C,y)}{b} dy$

By arc length, $ds = \sqrt{1+(\frac{dx}{dy})^2} dy = \sqrt{1+(a/b)^2} dy$

So $u = \frac{1}{\sqrt{b^2+a^2}} \int f ds+C_1$

7. Solve $u_x+2u_y+(2x-y)u = 2x^2+3xy-2y^2$

Use coordinate method, (Lagrange's method is long and tedious)

$x' = x+2y,y'=2x-y,u_{x} = u_{x'}+2u_{y'},u_{y} = 2u_{x'}-u_{y'}$

$x' = <1,2>, y'=<2,-1>, 2x'-y'=<2,4>-<2,-1>=<0,5> = 5y$

$y = \frac{2}{5} x' - \frac{1}{5} y', x = x' - (\frac{4}{5} x' - \frac{2}{5} y') = \frac{1}{5} x' + \frac{2}{5} y'$

The PDE is converted to 

$u_{x'}+2u_{y'} +(2x-y)u = u_{x'}+2u_{y'}+ 2(2u_{x'}-u_{y'}) + (2(\frac{1}{5} x' + \frac{2}{5} y')-(\frac{2}{5} x' - \frac{1}{5} y'))u = 2(\frac{1}{5} x' + \frac{2}{5} y')^2+3(\frac{1}{5} x' + \frac{2}{5} y')(\frac{2}{5} x' - \frac{1}{5} y')-2(\frac{2}{5} x' - \frac{1}{5} y')^2 $

After simple algebra,

$$ (5u_{x'}+ y'u) = x'y'$$

And then it is reduced to a calculus 4 question
\end{document}
